\documentclass{article}

\textheight = 24.5cm	% alto texto impreso
\textwidth = 16.5cm	    % ancho texto impreso
\topmargin = -2cm	    % margen superior 3-2=1cm
\oddsidemargin = -0.5cm % margen izquierdo 4.5-0.5=4cm


% OTROS PAQUETES
\usepackage{enumerate}          % enumerados
\usepackage[utf8]{inputenc}     % acentos sin codigo
\usepackage{subfigure}          % subgraficos
\usepackage{adjustbox}          % Ajustar tamaño de tablas (Figuras dentro de numeraciones)
\usepackage{caption}            % Caption de las figuras
\usepackage{multicol}           % columnas
\usepackage{amsmath}    % paquete para usar \begin{equation}
\usepackage{amsthm}		% Definiciones y teoremas
\newtheorem{defi}{Definition}

%% Numeracion con formato: Page 3 of 20
\usepackage{fancyhdr}
\usepackage{lastpage}

\pagestyle{fancy}
\fancyhf[]{}

\chead{Apunte}  %titulos de la página.
\rfoot{Página \thepage \hspace{1pt} de \pageref{LastPage}}

\renewcommand{\headrulewidth}{1pt}  % Grosor de la línea del encabezado.
\renewcommand{\footrulewidth}{1pt}  % Grosor de la línea del pie de página. 
%% End. Numeracion con formato.

\usepackage{graphicx}
\graphicspath{images/}
\usepackage{blindtext}
\usepackage{subfiles} % Best loaded last in the preamble

\title{Subfiles package example}
\author{Overleaf}
\date{ }