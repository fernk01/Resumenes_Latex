\documentclass[../main.tex]{subfiles}

\begin{document} %%%%%%%%%%%%%%%%%%%%%%%%%%%%%%%%%%%%%%%%%%%%%%%%%%%%%%%%%%%%
\section{Big data y Almacenamiento Distribuido}
\subsection{Big Data}
    \textit{BIg Data} datos que no pueden ser procesados por métodos tradicionales. 
    
    Se caracterizan por las 5 \textit{V's}:
    \begin{itemize}
        \item Volumen: cantidad de datos.
        \item Velocidad: velocidad a la que se generan los datos.
        \item Variedad: variedad de datos.
        \item Veracidad: calidad de los datos.
        \item Valencia: valor de los datos.
    \end{itemize}

    \subsubsection{Volumen}
        El impedimento de las computadoras a manejar grandes catidades de datos.

        Cluster: un conjunto de computadoras que trabajan enconjunto y pueden ser vistas como un sistema único.

    \subsubsection{Velocidad}
        La velocidad con la cual se generan los mismos.
        Para procesar los datos se necesita que el tiempo de procesamiento sea menor al tiempo de generación de los datos, sino estos se aculan y el algoritmo puede descartar una gran volumen de datos.



\section{VIDEO}
    \begin{definition} \textbf{(Map-Reduce)}
        Procesamiento distribuido de datos utilizando un cluster.
        \begin{itemize}
            \item Modelo de programación para procesar grandes volúmenes de datos.
            \item Sugue de la necesidad de procesar grandes volúmenes de datos de forma escalable.
        \end{itemize}

        \begin{itemize}
            \item El usuario especifica una funcion una función \textbf{map} que procesa un par clave/valor para generar un conjunto de pares clave/valor intermedios.
            \item Se debe especificar una función \textbf{reduce} que combina todos los valores asociados a una misma clave intermedia.
        \end{itemize}
    \end{definition}

    \begin{definition} \textbf{(Map)}
        \begin{itemize}
            \item Transfroma nuestros datos.
            \item Debe ser aplicada a cada dato de nuestro ser.
            \item Puede ser paralelizada y distribuirse entre las distintas máquinas de un cluster.
        \end{itemize}
        
        Algunas diferencias dependientes de la implementación:
        \begin{itemize}
            \item \textbf{Hadoop:} $Map(k,v) \rightarrow list(k2,v2)$
            \item \textbf{spark:} $Map(r) \rightarrow list(r')$
        \end{itemize}
    \end{definition}

    \begin{definition} \textbf{(Reduce)}
        \begin{itemize}
            \item Combina los resultados del map.
            \item Es necesario procesar los datos de todas las máquinas del cluster.
            \item Reduce locales en paralelo y reduce entre máquinas mediente esta de shuffle \& sort.
        \end{itemize}

        Algunas diferencias dependientes de la implementación:
        
        \underline{\textbf{Hadoop:}} $RecudeByKey((k, v), f) \rightarrow list(k, v)$
        \begin{itemize}
            \item El sistema agrupa todos los registros para los cuales la clave es la misma.
            \item Requiere que todos los registros de igual clave estén en la misma máquina que ejecute el reduce: Shuffle \& Sort.
        \end{itemize}

        \underline{\textbf{Spark:}}
        \begin{itemize}
            \item La función reduce toma dos valores para dar como resultado la combinación de ambos.
            \item El resultado de un reduce estre dos registros es un input del siguiente reduce.
            \item Opereaciones \textbf{conmutativas} y \textbf{asociativas} de modo de poder ejecutarse distribuidas.
        \end{itemize}
    
    \end{definition}

    \begin{definition} \textbf{(Cluster)}
        Conjunto de computadoras que trabajan juntas y pueden ser vistas como un sistema único.
    \end{definition}

    \begin{definition} \textbf{(Almacenamiento distribuido)}
        \begin{itemize}
            \item FIleSystem Distribuido (DFS)
            \item Encargado de gastionar cómo y dónde guardar la información en una computadora, y cómo poder consultarla.
            \item Almacenar grandes volúmenes de datos en multiples equipos.
            \item Replicación de datos para tolerancia a fallos.
            \item Tolerancia a fallos.
            \item Alta disponibilidad (seguir funcionando aunque un nodo falle).
            \item Relativo a bajo costo.
        \end{itemize}
        
        \underline{Ejemplos de SD:}
        \begin{itemize}
            \item GDS (Google File System)
            \item HDFS (Hadoop Distributed File System)
            \item CEPH (Ceph File System)
            \item S3 (Amazon Simple Storage Service)
        \end{itemize}
    \end{definition}


















\end{document}  %%%%%%%%%%%%%%%%%%%%%%%%%%%%%%%%%%%%%%%%%%%%%%%%%%%%%%%%%%%%%