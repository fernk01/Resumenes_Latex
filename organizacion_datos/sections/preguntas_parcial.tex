\documentclass[../main.tex]{subfiles}

\begin{document} %%%%%%%%%%%%%%%%%%%%%%%%%%%%%%%%%%%%%%%%%%%%%%%%%%%%%%%%%%%%
\section{Preguntas Tipo parcial}
    \begin{enumerate}
        \item El NPL sirve para\dots
            \begin{itemize}
                \item detertar emociones
                \item crear motor de busqueda.
                \item traducir texto
                \item \textbf{todas las anteriores} \checkmark
            \end{itemize}
        \item Count Vectorizer y TF-IDF siempre se guardan todas las palabras de un cuerpoz.
            \begin{itemize}
                \item \textbf{Falso} \checkmark
                \item Verdadero
            \end{itemize}
        \item Si coseno del angulo me da 0, significa que:
            \begin{itemize}
                \item El vector es exactamente el mismo.  
                \item \textbf{Los vectores son perpendiculares}. \checkmark
                \item Los vectores son paralelos pero su norma puede ser distinta.
                \item Los vectores son paralelos y su norma es la misma.
            \end{itemize}
        \item TF-IDF siempre de mejores resultados que Count Vectorizer.
            \begin{itemize}
                \item Verdadero
                \item \textbf{Falso} \checkmark
            \end{itemize}
        \item El TF-IDF es una forma de darle importancia a la "raridad" de una palabra en el cuerpo que se hace la busqueda.
            \begin{itemize}
                \item \boxed{\textbf{Verdadero} \checkmark}
                \item Falso
            \end{itemize}
        \item Stemming necesita un vocavulario de ante mano para llevar las palabras a sus raices.
            \begin{itemize}
                \item Verdadero
                \item \textbf{Falso} \checkmark
            \end{itemize}
        \item Lemmatization puede relacionar dos palabras que a lo mejor no tienen exactamente el mismo significado.
            \begin{itemize}
                \item \textbf{Verdadero} \checkmark
                \item Falso
            \end{itemize}
    \end{enumerate}

\end{document}  %%%%%%%%%%%%%%%%%%%%%%%%%%%%%%%%%%%%%%%%%%%%%%%%%%%%%%%%%%%%%