\documentclass[../main.tex]{subfiles}

\begin{document} %%%%%%%%%%%%%%%%%%%%%%%%%%%%%%%%%%%%%%%%%%%%%%%%%%%%%%%%%%%%

\section{Introducción  a la ciencia de datos.}
    Tipos de variables:
    \begin{enumerate}
        \item Variables Independientes (entradas)
            \begin{enumerate}
                \item Cualitativas
                    \begin{enumerate}
                        \item Texto
                            \begin{enumerate}
                                \item Nominales (categorías, Por ejemplo: países, sexo)
                                \item Ordinales (poco, mucho, muchísimo, Por ejemplo: nivel de tabaquismos)
                                
                            \end{enumerate}
                        \item Númericas
                            \begin{enumerate}
                                \item Nominales
                                \item Ordinales
                            \end{enumerate}
                    \end{enumerate}
                \item Cuantitativas: cuando hablamos de cantidad
                    \begin{enumerate}
                        \item Discretas: Por ejemplo año, mes, edad, etc.
                        \item Continuas: Por ejemplo altura, peso, etc.
                    \end{enumerate}
            \end{enumerate}
        \item Variables Dependientes (salidas, categorías)
    \end{enumerate}
	
    Las variables \textbf{cualitativas} son aquellas que describen características o cualidades y no pueden ser medidas en términos numéricos. Las variables \textbf{cuantitativas}, por otro lado, son aquellas que pueden ser medidas en términos numéricos y tienen valores numéricos.\\
		
    Las \textbf{variables nominales} son aquellas que no tienen un orden natural, como el género o el color de ojos. Las \textbf{variables ordinales} son aquellas que tienen un orden natural, como el nivel de educación (primaria/secundaria/universidad), nivel de tabaquismos: Clasificamos como leve (1), moderado (2), nivel medio (3), importante (4) y muy importante (5).\\
		 
     \textbf{Variables y tipos de problemas (video 02a min 9:00)}
     \begin{enumerate}
        \item 
            Si la variable dependiente es \textbf{cualitativa}, el tipo de problema es de \textbf{clasificación}. 
        \item 
            Si la variable dependiente es \textbf{cuantitativa}, el problema es de \textbf{regresión}.
            Por ejemplo si quiero predecir el precio de una propiedad.
        \item 
            Si \textbf{NO hay variable} dependientes, el problema es agrupamiento.
     \end{enumerate}
	 
    La variable pueden presentar:
    \begin{enumerate}
        \item Outliers (valores atípicos). Pueden ser erroreso un dato que se sale de la norma.
        \item Correlación:
            \begin{enumerate}
                \item Positiva
                \item Negativa
                \item Sin correlación.
            \end{enumerate}
        \item Varianza
        \item Covarianza
    \end{enumerate}

	
\end{document}  %%%%%%%%%%%%%%%%%%%%%%%%%%%%%%%%%%%%%%%%%%%%%%%%%%%%%%%%%%%%%