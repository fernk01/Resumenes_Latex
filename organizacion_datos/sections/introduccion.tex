\documentclass[../main.tex]{subfiles}

\begin{document} %%%%%%%%%%%%%%%%%%%%%%%%%%%%%%%%%%%%%%%%%%%%%%%%%%%%%%%%%%%%

\section{Introducción al Machine Learning.}

    En el contexto del aprendizaje supervisado, se trabaja con \textit{variables independientes} (también conocidas como características, predictores o variables de entrada) y \textit{variables dependientes} (también llamadas variables objetivo o target o variables de salida). Las variables independientes son las que se utilizan para hacer predicciones o estimaciones, mientras que las variables dependientes son los resultados que se intentan predecir o modelar. Por ejemplo, en un problema de predicción de precios de viviendas, las características como el número de habitaciones, la ubicación y el tamaño de la propiedad serían las variables independientes, mientras que el precio de venta sería la variable dependiente.

    \begin{itemize}
        \item Variables Independientes (entradas)
            \begin{itemize}
                \item Cualitativas
                    \begin{itemize}
                        \item Texto
                            \begin{itemize}
                                \item Nominales (categorías, Por ejemplo: países, sexo)
                                \item Ordinales (poco, mucho, muchísimo, Por ejemplo: nivel de tabaquismos)
                                
                            \end{itemize}
                        \item Númericas
                            \begin{itemize}
                                \item Nominales
                                \item Ordinales
                            \end{itemize}
                    \end{itemize}
                \item Cuantitativas: cuando hablamos de cantidad
                    \begin{itemize}
                        \item Discretas: Por ejemplo año, mes, edad, etc.
                        \item Continuas: Por ejemplo altura, peso, etc.
                    \end{itemize}
            \end{itemize}
        \item Variables Dependientes (salidas, categorías)
    \end{itemize}
	
    \begin{definition}\textbf{(variables cualitativas)} 
        Son aquellas que describen características o cualidades y no pueden ser medidas en términos numéricos. Las variables \textbf{cuantitativas}, por otro lado, son aquellas que pueden ser medidas en términos numéricos y tienen valores numéricos.
    \end{definition}
        
    \begin{definition}\textbf{variables nominales}
        Son aquellas que no tienen un orden natural, como el género o el color de ojos. Las \textbf{variables ordinales} son aquellas que tienen un orden natural, como el nivel de educación (primaria, secundaria, universidad), nivel de tabaquismos: Clasificamos como leve (1), moderado (2), nivel medio (3), importante (4) y muy importante (5).
    \end{definition}

	 
     \underline{Variables y tipos de problemas (video 02a min 9:00)}
     \begin{enumerate}
        \item 
            Si la variable dependiente es \textbf{cualitativa}, el tipo de problema es de \textbf{clasificación}. 
        \item 
            Si la variable dependiente es \textbf{cuantitativa}, el problema es de \textbf{regresión}.
            Por ejemplo si quiero predecir el precio de una propiedad.
        \item 
            Si \textbf{NO hay variable} dependientes, el problema es agrupamiento.
     \end{enumerate}
	 
     \begin{itemize}
        \item \textbf{Outliers:} Valores atípicos, pueden ser errores o un dato que se sale de la norma.
        \item \textbf{Correlación:} 
            \begin{itemize}
                \item \textbf{Positiva:} Cuando una variable aumenta la otra también.
                \item \textbf{Negativa:} Cuando una variable aumenta la otra disminuye.
                \item \textbf{Sin correlación:} Cuando una variable aumenta la otra no cambia.
            \end{itemize}
        \item \textbf{Varianza:} Es la medida de dispersión de una variable respecto a su media. Si la varianza es alta, los datos están muy dispersos, mientras que si la varianza es baja, los datos están muy agrupados.
        \item \textbf{Covarianza:} Es una medida de la relación lineal entre dos variables aleatorias. Indica cómo varían conjuntamente dos variables aleatorias respecto a sus medias. Si la covarianza es positiva, las variables aumentan o disminuyen conjuntamente, mientras que si la covarianza es negativa, una variable aumenta mientras la otra disminuye.
    \end{itemize}

	% Cargamos una imagen
    \begin{figure}[htb]
        \centering
        \includegraphics[scale=0.4]{./images/machine.jpg}
        \caption{Metodología de Machine Learning.}
        \label{fig:figura1}
    \end{figure}
\end{document}  %%%%%%%%%%%%%%%%%%%%%%%%%%%%%%%%%%%%%%%%%%%%%%%%%%%%%%%%%%%%%