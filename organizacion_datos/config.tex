\documentclass{article}
%%  Configuracion de la pagina.
\usepackage[T1]{fontenc}
\usepackage[spanish]{babel}
%%
%% Dimensiones de la pagina.
% \textheight = 24cm	% alto texto impreso
% \textwidth = 16cm	    % ancho texto impreso
% \topmargin = -2cm	    % margen superior 3-2=1cm
% \oddsidemargin = -0.5cm % margen izquierdo 4.5-0.5=4cm
%%
\usepackage{graphicx} % Required for inserting images
%%  
\usepackage{subfiles} % Best loaded last in the preamble
%%  Para tener una bibliografia.
\usepackage{csquotes}
\usepackage{biblatex} %Imports biblatex package
\addbibresource{bibliography.bib} %Import the bibliography file
\nocite{*}
%%
%% Numeracion con formato: Page 3 of 20
\usepackage{fancyhdr}
\usepackage{lastpage}

\pagestyle{fancy}
\fancyhf[]{}

\chead{Apunte de Organicación de datos FIUBA}  %titulos de la página.
\rfoot{Página \thepage \hspace{1pt} de \pageref{LastPage}}

\renewcommand{\headrulewidth}{1pt}  % Grosor de la línea del encabezado.
\renewcommand{\footrulewidth}{1pt}  % Grosor de la línea del pie de página. 
%%
%% Referecias a las secciones.
\usepackage{hyperref}
%%
%% Configuracion para el indice.
\hypersetup{
    colorlinks=true,
    linkcolor=blue,
    filecolor=magenta,      
    urlcolor=cyan,
    pdftitle={Overleaf Example},
    pdfpagemode=FullScreen,
    }  
\urlstyle{same}
%%
%% para usar icono de checkmark
\usepackage{tikz}
\def\checkmark{\tikz\fill[scale=0.4](0,.35) -- (.25,0) -- (1,.7) -- (.25,.15) -- cycle;} 
%% Cambiar colo a las section y subsection
\usepackage{xcolor}
\usepackage{sectsty}
\chapterfont{\color{blue}}  % sets colour of chapters
\sectionfont{\color{blue}}  % sets colour of sections
\subsectionfont{\color{purple}}  % sets colour of sections
\subsubsectionfont{\color{brown}}  % sets colour of sections
%% 
\hbadness=10000 % No warnings for underfull or overfull boxes (ignora este error en visual studio code)
%%  Titulo para la portada.
\title{Apunte Organicacion de datos}
\author{lcondoriz }
\date{May 2023}

%% Definiciones
\usepackage{amsthm}
\newtheorem{theorem}{Theorem}
\theoremstyle{definition}
\newtheorem{definition}{Definición}[section]

\newtheorem{example}{Ejemplo}[section]

% matematicas equaciones
\usepackage{amsmath}
\usepackage{mathtools} % para bmatrix*

% para cagas a color en los teoremas
\usepackage{tcolorbox}          % Cajas de colores.
\tcbuselibrary{theorems}
\newtcbtheorem[number within=section]{teorema}{Teorema}{}{}

%% Cuadros de texto sinopticos
\usepackage{schemata} 

\usepackage{amssymb} % para el simbolo de los reales

