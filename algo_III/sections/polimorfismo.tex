\documentclass[../main.tex]{subfiles}

\begin{document} %%%%%%%%%%%%%%%%%%%%%%%%%%%%%%%%%%%%%%%%%%%%%%%%%%%%%%%%%%%%
\section{Polimorfismo: los objetos se comportan a su manera}
    \subsection{¿Qué es polimorfismo?}
        El polimorfismo es la capacidad que tienen distintos objetos de responder de maneras diferentes a un mismo mensaje

        \begin{definition} \textbf{{Polimorfismo}}
            Llamamos polimorfismo a la posibilidad de que distintos objetos respondan de manera diferente ante la llegada del mismo mensaje.
            
            El polimorfismo es la capacidad que tienen distintos objetos de responder de maneras diferentes a un mismo mensaje.
        \end{definition}

        \begin{definition} \textbf{(mensaje polimorfo)}
            Un mensaje es polimorfo cuando la respuesta al mismo puede ser diferente en función del objeto receptor.
        \end{definition}

    \subsection{Polimorfismo y herencia: ¿realmente deben ir juntos?}
        El vínculo entre polimorfismo y herencia es, simplemente, una cuestión de implementación

    \subsection{Métodos abstractos y comprobación estática}
        En definitiva, los métodos abstractos sirven, en los lenguajes de comprobación dinámica, como un medio para obligar a las clases descendientes a implementar ese comportamiento. En los lenguajes de comprobación estática, hay ocasiones en que no podemos sino definir ciertos métodos abstractos si queremos que funcione el polimorfismo.

    \subsection{Polimorfismo sin herencia en lenguajes de comprobación estática: interfaces}
    Polimorfismo sin herencia. Java es uno de ellos, y el mecanismo que utiliza se denomina interfaces.

    Otra manera de ver a una interfaz es simplemente como un conjunto de métodos abstractos



















\end{document}  %%%%%%%%%%%%%%%%%%%%%%%%%%%%%%%%%%%%%%%%%%%%%%%%%%%%%%%%%%%%%
