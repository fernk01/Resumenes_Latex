\documentclass[../main.tex]{subfiles}

\begin{document} %%%%%%%%%%%%%%%%%%%%%%%%%%%%%%%%%%%%%%%%%%%%%%%%%%%%%%%%%%%%

\section{Parciales}
\textbf{¿Que es el stack ? Explique el mecanismo de funcionamiento del stack para x86 de la siguiente funcion int read(void *buff, size\_t num, int fd);.Como se pasan los parametros, direccion de retorno?.}\\

El stack o pila es una estructura de datos que almacena información de forma temporal y ordenada, siguiendo el principio LIFO (Last In, First Out), es decir, el último en entrar es el primero en salir. El stack se usa para guardar los datos locales de una función, las direcciones de retorno de las llamadas a funciones y los parámetros que se pasan a las funciones.\\

Para la función read(void *buff, size\_t num, int fd), que lee num bytes del archivo identificado por fd y los almacena en el buffer buff, se puede usar el stack para pasar los parámetros de la siguiente manera:
\begin{itemize}
    \item Se empujan los parámetros al stack en orden inverso, es decir, primero fd, luego num y finalmente buff.
    \item Se llama a la función read con la instrucción call, que empuja la dirección de retorno al stack y salta a la etiqueta de la función.
    \item Dentro de la función read, se accede a los parámetros usando el registro ebp (base pointer) como referencia. El registro ebp se usa para guardar el valor del registro esp (stack pointer) al entrar en la función, y así poder acceder a los parámetros y variables locales sin importar cómo cambie el esp durante la ejecución de la función
    \item Se usa la convención cdecl para limpiar el stack después de la llamada a la función. Esta convención establece que el código que llama a la función es responsable de restaurar el esp al valor que tenía antes de empujar los parámetros. Esto se hace sumando al esp el tamaño total de los parámetros.
\end{itemize}

Un posible código en ensamblador x86 para este ejemplo sería:

; Código que llama a la función read ; Supongamos que fd = 3 (stdin), num = 100 y buff apunta a una zona de memoria reservada mov eax, 3 ; fd push eax ; empujar fd al stack mov eax, 100 ; num push eax ; empujar num al stack mov eax, buff ; buff push eax ; empujar buff al stack call read ; llamar a la función read add esp, 12 ; limpiar el stack (3 parámetros de 4 bytes cada uno)

; Código de la función read read: push ebp ; guardar el valor anterior de ebp mov ebp, esp ; copiar el valor de esp a ebp ; Ahora los parámetros se pueden acceder como [ebp+8], [ebp+12] y [ebp+16] ; Aquí iría el código para leer del archivo y escribir en el buffer ; usando las instrucciones syscall o int 80h mov esp, ebp ; restaurar el valor de esp pop ebp ; restaurar el valor de ebp ret ; retornar a la dirección guardada en el stack

\end{document}  %%%%%%%%%%%%%%%%%%%%%%%%%%%%%%%%%%%%%%%%%%%%%%%%%%%%%%%%%%%%%