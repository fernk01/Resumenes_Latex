\documentclass[../main.tex]{subfiles}

\begin{document} %%%%%%%%%%%%%%%%%%%%%%%%%%%%%%%%%%%%%%%%%%%%%%%%%%%%%%%%%%%%
\section{Introducción: x86}

\subsection{Ley de Moore}
    En 1965 Gordon Moore, cofundador de Intel formuló una ley empírica que se ha podido constatar hasta nuestros días que dice:\\

    \textit{”Aproximadamente cada dos años se duplica el número de transistores en un microprocesador por unidad de área”}

\subsection{Arquitectura x86: Hardware}
    \underline{Registros de Segmento}
    \begin{itemize}
        \item \textbf{CS}: Segmento de código.
        \item \textbf{DS}: Segmento de datos.
        \item \textbf{SS}: Segmento de pila.
        \item \textbf{ES}: Segmento extra.
    \end{itemize}

    \subsubsection{El Stack}
        El stack o pila es una estructura de datos que almacena información de forma temporal y ordenada, siguiendo el principio LIFO. El stack se usa
        para guardar los datos locales de una función, las direcciones de retorno de las llamadas a funciones y los parámetros que se pasan a las funciones. Más precisamente la ejecución de un programa se basa practicamente en pusher.

    \subsubsection{Estructura del Stack Frame}
        En la arquitectura x86 los programas utilizan el stack del programa para soportar la llamada a funciones (o procedimientos). La máquina utiliza el stack para:
        
        \begin{itemize}
            \item Almacenar los parámetros de la función.
            \item Almacenar las variables locales de la función.
            \item Almacenar el valor de retorno de la función.
            \item Almacenar los registros que se deben preservar.
        \end{itemize}


\end{document}  %%%%%%%%%%%%%%%%%%%%%%%%%%%%%%%%%%%%%%%%%%%%%%%%%%%%%%%%%%%%%