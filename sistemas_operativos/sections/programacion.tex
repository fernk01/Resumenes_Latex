\documentclass[../main.tex]{subfiles}

\begin{document} %%%%%%%%%%%%%%%%%%%%%%%%%%%%%%%%%%%%%%%%%%%%%%%%%%%%%%%%%%%%
\section{Programación} 
    \begin{enumerate}
        \item \textbf{fork():}
        \item \textbf{getpid()}
    \end{enumerate}
    \subsection{Fork}
        La única forma de que un usuario cree un proceso en el sistema operativo UNIX es llamando a la system call fork.

        Crea un proceso y devuelve su id.\\

        ¿Que hace fork?:
        \begin{itemize}
            \item Crea y asigna una nueva entrada en la \textbf{Process Table} para el nuevo proceso.
            \item Asigna un número de ID único al proceso hijo.
            \item Crea una copia lógica del \textbf{contexto} del proceso padre, algunas de esas partes pueden ser compartidas como la sección text.
            \item Realiza ciertas operaciones de I/O.
            \item Devuelve el número de ID del hijo al proceso padre, y un 0 al proceso hijo.
        \end{itemize}

        \underline{Código de error:}
        \begin{itemize}
            \item \textbf{EAGAIN:} No hay suficientes recursos disponibles para crear un nuevo proceso.
            \item \textbf{ENOMEM:} No hay suficiente memoria disponible para crear un nuevo proceso.
            \item \textbf{EPERM:} El proceso no tiene permiso para crear un nuevo proceso.
        \end{itemize}


\end{document}  %%%%%%%%%%%%%%%%%%%%%%%%%%%%%%%%%%%%%%%%%%%%%%%%%%%%%%%%%%%%%