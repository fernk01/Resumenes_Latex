\documentclass[../main.tex]{subfiles}

\begin{document} %%%%%%%%%%%%%%%%%%%%%%%%%%%%%%%%%%%%%%%%%%%%%%%%%%%%%%%%%%%%
\section{Pilas} 
        
    Una pila es una estructura de datos que permite almacenar y recuperar datos, siendo el modo de acceso a sus elementos de tipo LIFO (del inglés Last In, First Out, «último en entrar, primero en salir»).

    \subsection{Operaciones}
        \begin{itemize}
            \item \textbf{Push:} Inserta un elemento en la pila.
            \item \textbf{Pop:} Elimina un elemento de la pila.
            \item \textbf{Top:} Devuelve el elemento que está en la cima de la pila.
            \item \textbf{Empty:} Devuelve un valor booleano indicando si la pila está vacía o no.
            \item \textbf{Size:} Devuelve el tamaño de la pila.
        \end{itemize}


        
\end{document}  %%%%%%%%%%%%%%%%%%%%%%%%%%%%%%%%%%%%%%%%%%%%%%%%%%%%%%%%%%%%%