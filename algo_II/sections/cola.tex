\documentclass[../main.tex]{subfiles}

\begin{document} %%%%%%%%%%%%%%%%%%%%%%%%%%%%%%%%%%%%%%%%%%%%%%%%%%%%%%%%%%%%
\section{Cola} 
        
    Una cola es una estructura de datos que permite almacenar y recuperar datos, siendo el modo de acceso a sus elementos de tipo FIFO (del inglés First In, First Out, «primero en entrar, primero en salir»).

    \subsection{Operaciones}
        \begin{itemize}
            \item \textbf{Push:} Inserta un elemento en la cola.
            \item \textbf{Pop:} Elimina un elemento de la cola.
            \item \textbf{Front:} Devuelve el elemento que está en la parte frontal de la cola.
            \item \textbf{Back:} Devuelve el elemento que está en la parte posterior de la cola.
            \item \textbf{Empty:} Devuelve un valor booleano indicando si la cola está vacía o no.
            \item \textbf{Size:} Devuelve el tamaño de la cola.
        \end{itemize}

        
\end{document}  %%%%%%%%%%%%%%%%%%%%%%%%%%%%%%%%%%%%%%%%%%%%%%%%%%%%%%%%%%%%%