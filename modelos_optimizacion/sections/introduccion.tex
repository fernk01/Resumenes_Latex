\documentclass[../main.tex]{subfiles}

\begin{document} %%%%%%%%%%%%%%%%%%%%%%%%%%%%%%%%%%%%%%%%%%%%%%%%%%%%%%%%%%%%
    \section{Introducción a la construción de modelos} 
        La investigación de operaciones (IO) es, simplemente, un enfoque científico en la toma de decisiones que busca el mejor diseño y operar un sistema.

        Por \textbf{sistema}, se quiere dar a entender una organización de componentes interdependientes, que trabajan juntos para lograr el objetivo del sistema.

        El en enfoque científico de toma de decisiones, se requiere el uso de uno o más \textbf{modelos matemáticos}.

        \subsection{Modelos prescriptivos o de optimización}
            Este tipo de modelos dicta el comportamiento para la organización que le permitirá alcanzar sus objetivos. 

            Elementos de un modelo de optimización:
            \begin{itemize}
                \item \textbf{Función objetivo:} es una función matemática de las variables de decisión que se debe maximizar o minimizar.
                \item \textbf{Variables de decisión:} son las variables que se pueden controlar para lograr los objetivos del sistema.
                \item \textbf{Restricciones:} son las limitaciones que se deben satisfacer.
            \end{itemize}

            Un modelo de optimización trata de encontrar valores, entre el conjunto de todos los valores para las variables de decisión, que optimicen (maximicen o minimicen) la función objetivo, y que satisfagan las restricciones.

        \subsection{Modelos estáticos y dinámicos}
            Un modelo estático es aquel que no considera el paso del tiempo. Un modelo dinámico, en cambio, sí lo hace.

            Un modelo dinámico puede ser \textbf{determinístico} o \textbf{estocástico}. Un modelo determinístico es aquel en el que todos los parámetros son conocidos. Un modelo estocástico es aquel en el que al menos un parámetro es desconocido.

\end{document}  %%%%%%%%%%%%%%%%%%%%%%%%%%%%%%%%%%%%%%%%%%%%%%%%%%%%%%%%%%%%%
