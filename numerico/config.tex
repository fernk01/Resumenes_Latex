\documentclass{article}
%%  Configuracion de la pagina.
\usepackage[T1]{fontenc}
\usepackage[spanish]{babel}
%%
%% Dimensiones de la pagina.
% \textheight = 24cm	% alto texto impreso
% \textwidth = 16cm	    % ancho texto impreso
% \topmargin = -2cm	    % margen superior 3-2=1cm
% \oddsidemargin = -0.5cm % margen izquierdo 4.5-0.5=4cm
%%
\usepackage{graphicx}
%%  
\usepackage{subfiles} % Best loaded last in the preamble
%%  Para tener una bibliografia.
\usepackage{csquotes}
\usepackage{biblatex} %Imports biblatex package
\addbibresource{bibliography.bib} %Import the bibliography file
\nocite{*}
%%
%% Numeracion con formato: Page 3 of 20
\usepackage{fancyhdr}
\usepackage{lastpage}

\pagestyle{fancy}
\fancyhf[]{}

\chead{Apunte Análisis Numérico - FIUBA}  %titulos de la página.
\rfoot{Página \thepage \hspace{1pt} de \pageref{LastPage}}

\renewcommand{\headrulewidth}{1pt}  % Grosor de la línea del encabezado.
\renewcommand{\footrulewidth}{1pt}  % Grosor de la línea del pie de página. 
%%
%% Referecias a las secciones.
\usepackage{hyperref}
%%
%% Configuracion para el indice.
\hypersetup{
    colorlinks=true,
    linkcolor=blue,
    filecolor=magenta,      
    urlcolor=cyan,
    pdftitle={Overleaf Example},
    pdfpagemode=FullScreen,
    }  
\urlstyle{same}
%%
%% para usar icono de checkmark
\usepackage{tikz}
\def\checkmark{\tikz\fill[scale=0.4](0,.35) -- (.25,0) -- (1,.7) -- (.25,.15) -- cycle;} 
%% Cambiar colo a las section y subsection
\usepackage{xcolor}
\usepackage{sectsty}
\chapterfont{\color{blue}}  % sets colour of chapters
\sectionfont{\color{blue}}  % sets colour of sections
\subsectionfont{\color{purple}}  % sets colour of sections
\subsubsectionfont{\color{brown}}  % sets colour of sections
%% 
\hbadness=100000 % No warnings for underfull or overfull boxes (ignora este error en visual studio code)
%%  Titulo para la portada.
\title{Apunte Análisis Numérico - FIUBA}
\author{lcondoriz }
\date{Jun 2023}
%%
%% INICIO: Octave Code
\usepackage{listings}
\usepackage{xcolor}

\definecolor{codegreen}{rgb}{0,0.6,0}
\definecolor{codegray}{rgb}{0.5,0.5,0.5}
\definecolor{codepurple}{rgb}{0.58,0,0.82}
\definecolor{backcolour}{rgb}{0.95,0.95,0.92}

\lstdefinestyle{mystyle}{
    backgroundcolor=\color{backcolour},   
    commentstyle=\color{codegreen},
    keywordstyle=\color{magenta},
    numberstyle=\tiny\color{codegray},
    stringstyle=\color{codepurple},
    basicstyle=\ttfamily\footnotesize,
    breakatwhitespace=false,         
    breaklines=true,                 
    captionpos=b,                    
    keepspaces=true,                 
    numbers=left,                    
    numbersep=5pt,                  
    showspaces=false,                
    showstringspaces=false,
    showtabs=false,                  
    tabsize=2
}
% Agregamos los acentos manualmente.
\lstset{
    breaklines=true,
    %
    extendedchars = true,
    literate =
    {á}{{\'a}}1 {é}{{\'e}}1 {í}{{\'i}}1 {ó}{{\'o}}1 {ú}{{\'u}}1
    {Á}{{\'A}}1 {É}{{\'E}}1 {Í}{{\'I}}1 {Ó}{{\'O}}1 {Ú}{{\'U}}1
    {à}{{\`a}}1 {è}{{\`e}}1 {ì}{{\`i}}1 {ò}{{\`o}}1 {ù}{{\`u}}1
    {À}{{\`A}}1 {È}{{\'E}}1 {Ì}{{\`I}}1 {Ò}{{\`O}}1 {Ù}{{\`U}}1
    {ä}{{\"a}}1 {ë}{{\"e}}1 {ï}{{\"i}}1 {ö}{{\"o}}1 {ü}{{\"u}}1
    {Ä}{{\"A}}1 {Ë}{{\"E}}1 {Ï}{{\"I}}1 {Ö}{{\"O}}1 {Ü}{{\"U}}1
    {â}{{\^a}}1 {ê}{{\^e}}1 {î}{{\^i}}1 {ô}{{\^o}}1 {û}{{\^u}}1
    {Â}{{\^A}}1 {Ê}{{\^E}}1 {Î}{{\^I}}1 {Ô}{{\^O}}1 {Û}{{\^U}}1
    {œ}{{\oe}}1 {Œ}{{\OE}}1 {æ}{{\ae}}1 {Æ}{{\AE}}1 {ß}{{\ss}}1
    {ç}{{\c c}}1 {Ç}{{\c C}}1 {ø}{{\o}}1 {å}{{\r a}}1 {Å}{{\r A}}1
    {€}{{\EUR}}1 {£}{{\pounds}}1 {ñ}{{\~n}}1,
}

\lstset{style=mystyle}
\renewcommand\lstlistingname{Script} % Cambia el nombre de caption
%% FIN: Octave Code