\documentclass[../main.tex]{subfiles}


\begin{document} %%%%%%%%%%%%%%%%%%%%%%%%%%%%%%%%%%%%%%%%%%%%%%%%%%%%%%%%%%%%
\section{Errores den los métodos numéricos} 
    \subsection{Errores de adsoluto y relativo}
    Supongamos que obtenemos de alguna forma (por ejemplo, una medición) cierto valor $\overline{m}$. Sabemos que el valor $\ll  exacto \gg $ de dicho valor es $m$.
    \begin{itemize}
        \item Error adsoluto: $|e_a| = |m - \overline{m}|$
        \item Error relativo: $|e_r| = \frac{|m - \overline{m}|}{|m|} = \frac{|e_a|}{|m|}$, (siempre que $m \neq 0$)
    \end{itemize}        

    \subsection{Condición de un problema}
        El primer caso, el análisis de la propagación de los errores inherentes, permite establecer si el problema está \textit{bien o mal condicionado}.
        \begin{itemize}
            \item \textbf{Bien condicionado:} Si al analizar un pequeño cambio (o perturbación)
            en los datos el resultado se modifica levemente (o tiene un pequeño cambio).
            \item \textbf{Mal condicionado:} Si, por el contrario, el resultado se modifica notablemente o se vuelve oscilante.
        \end{itemize}

        Si está mal condicionado, no hay forma de corregirlo cambiando el algoritmo pues el problema está en el modelo matemático.

        \begin{theorem}
            Let \(f\) be a function whose derivative exists in every point, then \(f\) 
            is a continuous function.
        \end{theorem}

        \begin{definition}[]
            Un problema matemático (numérico) se dice que está \textbf{bien condicionado} si pequeñas variaciones en los datos de entrada se traducen en pequeñas variaciones de los resultados.
        \end{definition}


    \subsection{Propagación de errores}
        Propagación de dos de los errores más problemáticos, el inherente y el de redondeo.


        \subsubsection{Propagación del error inherente}
            \begin{equation}
                e_{y_i} = \sum_{j=1}^{n}  \frac{\partial y_i(\widetilde{x}) }{\partial x_j} \cdot e_{x_j} \quad  \quad para \quad i = 1,2,...,m
            \end{equation}

            \begin{enumerate}
                \item \textbf{Suma:} Si $y(x_1, x_2) = x_1 + x_2$, entonces
                    \begin{equation}
                        e_y = e_{x_1} + e_{x_2} = \frac{\partial y(\widetilde{x_1}, \widetilde{x_2})}{\partial x_1} \cdot e_{x_1} + \frac{\partial y(x_1, x_2)}{\partial x_2} \cdot e_{x_2} 
                    \end{equation}
            \end{enumerate}
        






\end{document}  %%%%%%%%%%%%%%%%%%%%%%%%%%%%%%%%%%%%%%%%%%%%%%%%%%%%%%%%%%%%%
